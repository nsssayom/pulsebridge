\section{Other Applications}
\label{sec:appl}

The safety island can be used, for obvious reasons, in contexts where system requirements include the combination of high-performance needs and safety requirements. However, most of its features can provide other type of services to HPC devices, such as, security and Reliability, Availability and Serviceability (RAS). Therefore, there is a broad area of application for the safety island beyond safety requirements strictly. For the sake of illustration, we introduce the use of the safety island for RAS and security applications in the remaining of this section.

\subsection{RAS}
HPC devices used for servers and supercomputers, to name some application domains, have strict RAS requirements. This relates to the relatively higher criticality of the applications run in those domains when compared with most desktop computers, as well as the much higher exposure to faults due to having very high occupancy and, in the context of supercomputers, typically thousands of computing nodes operating in parallel cooperatively. For instance, in the case of a supercomputer where parallel applications required on average 1,000 CPUs, acceptable failure rates would decrease at least by a factor of 1,000 with respect to single-CPU computers.

Components such as the SafeSU or watchdogs can provide error detection capabilities able to trigger recovery actions at software level. Analogously, logging features can be used to assist recovery by providing relevant information about the error detected, or can also be used to anticipate unrecoverable errors by monitoring recoverable ones. For instance, permanent faults can be detected by diagnosing error location with specific logging capabilities so that appropriate actions (e.g., CPU replacement in a supercomputer) can be taken before permanent faults lead to any unrecoverable error.


\subsection{Security}
It is well-known that all safety-critical systems are also security-critical. This relates to the fact that unintended failures in safety-critical systems could be produced intendedly by an attacker, hence creating at least similar risks. The opposite, instead, is not true, and systems can be security-critical but not safety-critical (e.g., a system managing personal information). 

A subset of the security concerns have analogous behavior to that of the safety concerns, being the only difference whether their root cause is intended (security concerns) or unintended (safety concerns). For instance, abuse in the access to shared resources may be caused by a faulty application or by a malicious attack. In both cases, the SafeSU could be leveraged to, at least, detect the abuse and take corrective actions. Tracing and logging features could be used to discern intended from unintended attacks based on their history or frequency of occurrence, for instance.

Security concerns may often require countermeasures to stop or fool attacks. Components such as the SafeTI may be leveraged for that purpose generating traffic that degrades the ability of the attacker to deduce information from the victim. For instance, in the case of side-channel attacks learning from memory access patterns of the victim, traffic can be injected making the attacker believe such traffic belongs to the victim so that wrong conclusions are reached (e.g., obtaining a wrong key or failing to narrow down enough the possibilities for the victim's key).

In any case, if the safety island is also used for security purposes, a number of additional security-specific technologies may need to be added to the safety (now safety-and-security) island, such as cryptographic accelerators, {\color{black}and means for detecting and defeating attacks.
Note that those security-related features focus on providing security services to the HPC device, whereas those discussed in Section~\ref{sec:seccons} focus on making safety island operation secure.}


