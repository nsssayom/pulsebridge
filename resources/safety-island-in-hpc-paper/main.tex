\documentclass[conference,a4paper]{IEEEtran}
\IEEEoverridecommandlockouts
% The preceding line is only needed to identify funding in the first footnote. If that is unneeded, please comment it out.
%% another package (only for this demo article)

\usepackage{physics}
\usepackage{cite}
\usepackage{amsmath,amssymb,amsfonts}
%\usepackage{algpseudocode, algorithm}
\usepackage{graphicx}
%\usepackage{subcaption}
\usepackage{textcomp}
\usepackage[dvipsnames,table,xcdraw]{xcolor}
\usepackage{cleveref}
\usepackage{multirow}
\usepackage{listings}
\usepackage{float}
\usepackage{fancyhdr}
\usepackage[table]{xcolor}
\usepackage{soul}
\usepackage{xurl} % fix the linebreaks for urls
\usepackage{colortbl}
\usepackage{tikz}

\newcommand*\boxednumber[1]{\tikz[baseline=(char.base)]{%
            \node[shape=rectangle,fill=black] (char) {\color{white}#1};}}

\usepackage{subfig}

\usepackage[ruled]{algorithm2e}

\usepackage{amsthm}

\definecolor{mGreen}{rgb}{0,0.6,0}
\definecolor{mGray}{rgb}{0.5,0.5,0.5}
\definecolor{mPurple}{rgb}{0.58,0,0.82}
\definecolor{mOrange}{rgb}{0.9,0.4,0.1}
\definecolor{backgroundColour}{rgb}{0.95,0.95,0.92}

\lstdefinestyle{CStyle}{
%    backgroundcolor=\color{backgroundColour},   
    backgroundcolor=\color{white},   
    commentstyle=\color{mGreen},
    keywordstyle=\color{magenta},
    numberstyle=\tiny\color{mGray},
    stringstyle=\color{mPurple},
    basicstyle=\scriptsize,
    breakatwhitespace=false,         
    breaklines=true,                 
    captionpos=b,                    
    keepspaces=true,                 
    numbers=left,                    
    numbersep=5pt,                  
    showspaces=false,                
    showstringspaces=false,
    showtabs=false,                  
    tabsize=2,
    frame=single,
    rulecolor=\color{black},
    language=C
}

%This is in the IEEE latext template. Do not remove it
\def\BibTeX{{\rm B\kern-.05em{\sc i\kern-.025em b}\kern-.08em
    T\kern-.1667em\lower.7ex\hbox{E}\kern-.125emX}}


\begin{document}

\title{Envisioning a Safety Island to Enable HPC Devices in Safety-Critical Domains}

\author{Jaume Abella$^1$, Francisco J. Cazorla$^1$, Sergi Alcaide$^1$, Michael Paulitsch$^2$, Yang Peng$^2$, In\^{e}s Pinto Gouveia$^2$
\vspace{0.2cm}\\
$^1$ Barcelona Supercomputing Center, Spain\\	
$^2$ Intel, Germany}


\maketitle

\begin{abstract}
HPC (High Performance Computing) devices increasingly become the only alternative to deliver the performance needed in safety-critical autonomous systems (e.g., autonomous cars, unmanned planes) due to deploying large and powerful multicores along with accelerators such as GPUs. However, the support that those HPC devices offer to realize safety-critical systems on top is heterogeneous.
%
Safety islands have been devised to be coupled to HPC devices and complement them to meet the safety requirements of an increased set of applications, yet the variety of concepts and realizations is large.

This paper presents our own concept of a safety island with two goals in mind: (1) offering a wide set of features to enable the broadest set of safety applications for each HPC device, and (2) being realized with open source components based on RISC-V ISA to ease its use and adoption. In particular, we present our safety island concept, the key features we foresee it should include, and its potential application beyond safety.
\end{abstract}



\section{Introduction}
\label{sec:intro}

HPC processors can deliver the performance needed for future embedded systems of autonomous cars and aircraft, which will rely heavily on performance-hungry Artificial Intelligence (AI) software, as well as a high number of processes to manage simultaneous events triggered by a plethora of sensors and by timers periodically. However, the support that those devices offer for their use in safety-related applications is different across devices, and so are the guarantees or additional support they need -- if any -- for their use in applications with varying needs in terms of integrity level, fail-safe or fail-operational needs, performance predictability, and time to recover from different types of errors.
 

In general, HPC processors can be used for fail-safe applications as long as a high-integrity microcontroller unit (MCU) is deployed along with the HPC processor, and the overall system architected so that the MCU can manage all errors affecting both, the HPC processor or the MCU itself, and guide the system to a safe state timely in accordance with application safety requirements. However, if the MCU cannot manage all errors in the HPC processor preserving safety requirements, or if the HPC processor must remain fault-tolerant to meet such safety requirements, then additional support is needed beyond that of the MCU. Such support can be part of the HPC device itself, or be delivered in the form of an enhanced MCU with extended safety capabilities to assist the HPC device.

To overcome the potential limitations that some HPC devices may offer for their use in safety-critical applications, \emph{safety islands} have been proposed recently~\cite{SiemensSI,IntelGo,NVIDIAdrive,NVIDIAdrive2}. While the term \emph{safety island} includes a heterogeneous set of devices, they normally offer two key sets of features: (1) a safe enclave to run safety-related applications, in the same way an MCU does. In fact, an MCU can be regarded as a safety island. However, safety islands offer lower performance than HPC devices, at least for a set of performance-demanding applications, and hence the need for an accompanying HPC device. (2) A safety island also offers specific support to manage an external device that needs to run safety-related functionalities. Such support may include watchdog services, performance monitoring capabilities, ability to initiate test capabilities in the other device or to test it externally, or support to orchestrate diverse redundant execution in the external device to name a few.

When considering the deployment of a safety island along with an HPC device, considerations about integration cannot be neglected. For instance, new cost-efficient chip production and packaging options (multi-die chips -- EMIB\footnote{Embedded Multi-Die Interconnect Bridge}, Foveros~\cite{packagingIntel}) provide a solution to deploy separate dies in the same package, which benefits a number of safety island requirements like fault independence with different dies while preserving high bandwidth communication.

This paper presents our own concept of a safety island, as well as our strategy to realize it based on open source RISC-V~\cite{RISCV} components with the aim of easing its use and adoption. 
In our case, we aim at providing a rich set of safety services to the HPC device to enable the deployment of safety-relevant applications on top with varying safety requirements, despite the potentially limited support that the HPC device may offer natively. Those services foreseen for our safety island include the following:
\begin{itemize}
\item Controllability features to configure the HPC device for fault management and containment, and for providing predictable performance.
\item Observability features for system validation, error diagnosis, and as the basis to build informed safety measures on top.
\item Safety measures to contain and manage errors, and to mitigate abnormal performance conditions.
\end{itemize}

In this paper we provide the following contributions for the design and deployment of our view of a safety island:
\begin{enumerate}
\item We present our concept of a safety island along with examples for its practical realization focusing on open source SoCs based on the RISC-V ISA.
\item We identify existing open source components providing controllability and observability features, and safety measures, appropriate for our Safety Island.
\item We identify further features to be developed in the future to complete and complement our safety island.
\item We devise further applications of the Safety Island beyond safety, such as security, reliability, and power and temperature management. To some extent, the foreseen security extensions are comparable to the ``Security Island'' already offered by some processors (e.g., Arm's TrustZone~\cite{TrustZone}). 
\end{enumerate}

The rest of the paper is organized as follows. Section~\ref{sec:back} provides some background and other solutions comparable or related to the safety island proposed in this paper. Section~\ref{sec:sais} presents our concept of safety island. Section~\ref{sec:comp} describes convenient components for our safety island, whether they already exist or not. Section~\ref{sec:appl} shows how the safety island could be used for other applications. Finally, Section~\ref{sec:concl} summarizes this paper.




\input{2.0.Background}
\input{3.0.SafetyIsland}
\section{Key Components and Technologies}
\label{sec:comp}

Realizing a safety island requires an SoC capable of executing safety-relevant functionalities, as well as capable of providing safety services to the HPC island. To build a functional and open source safety island, we identify a number of existing and under development components and technologies that need to be consistently integrated to form the safety island. Some of those components and technologies are introduced in this section.

\subsection{Baseline MPSoC}
As part of the H2020 SELENE project, an open source RISC-V based MPSoC suitable for the space, automotive and railway domains has been released~\cite{SELENEpaper}. The SELENE SoC offers a 6-core multicore based on Gaisler's NOEL-V cores~\cite{NOELV} and other GPL IPs~\cite{GRLIB}. Moreover, it includes a wide subset of the IPs described in the remaining of this subsection that make it further appropriate as the starting point to develop a safety island.

{\color{black}
However, there are other alternatives. Unfortunately, high-performance RISC-V cores are mostly proprietary, such as SiFive's P650 and others.
Some open source cores have been recently compared~\cite{dorflinger2021comparative}, including Rocket~\cite{asanovic2016rocket}, BOOM~\cite{celio2019broom}, CVA6~\cite{COREV}, and SHAKTI~\cite{gala2016shakti} C-Class implementations. No core is proven superior to the others in all fronts with varying conclusions for both ASIC and FPGA realizations if we consider performance, power efficiency, area, or maintainability. 
}


\subsection{Multicore Interference Monitors}
A key safety service in multicores relates to monitoring the interference across cores or other type of devices (e.g., accelerators) since such interference may affect real-time guarantees for safety-critical real-time tasks. Recently, the Safe Statistics Unit (SafeSU)~\cite{SafeSU,SafeSU2} has been proposed. It provides capabilities to monitor the traffic in AMBA interfaces such as AHB and AXI4, although its design has been made modular to enable its porting to other interfaces. The SafeSU allows measuring the interference each master device causes on each other device in different interfaces, and has been successfully integrated in the SELENE SoC~\cite{SELENEQoS}.
It remains to be studied how to tailor it to monitor the traffic in remote interconnects rather than those in the safety island itself.


\subsection{Multicore Interference Quotas}
The SafeSU~\cite{SafeSU,SafeSU2} has also been equipped with an interference control mechanism building on its interference monitoring capabilities. In particular, the SafeSU allows programming interference quotas that, upon being exceeded, trigger interrupts that can be immediately captured by the hypervisor or RTOS so that any action needed can be taken, in accordance with system needs (e.g., dropping the offending task, stalling it for a while, increasing QoS guarantees for the offended task). These interrupts have been properly connected to the corresponding interrupt controller at hardware level and successfully captured by the operating system on top, so the integration of the SafeSU with the software layers is simple.


\subsection{Performance Validation}
While monitoring and quota features during operation are key features needed for the design of the system, such system must be thoroughly tested to guarantee that timing overruns will not occur making deadline violation risk residual. Software tests provide limited controllability to exercise all performance corners since multicore interference scenarios can only be induced indirectly and, in some cases, without synchronous control. For instance, some traffic with long bursts can only be produced by devices such as Ethernet ports of the Direct Memory Access (DMA) controller, which are too hard to synchronize with traffic produced by the computing cores. To tackle this issue, the Safe Traffic Injector (SafeTI)~\cite{SafeTI} has been recently proposed. It allows programming arbitrary traffic patterns, including delays between consecutive transactions, fully synchronously, and allowing to generate any type of traffic including read/write, with arbitrary data transfer sizes, with/without burst behavior, etc., including repeated traffic, as well as fixed-size and infinite traffic patterns.
As for the SafeSU, it remains to be studied how to tailor the SafeTI to inject traffic in the HPC island from the safety island.


\subsection{Diverse Redundancy for Cores}
Functionalities with the highest integrity level (e.g., ASIL-D in automotive) require diverse redundancy in several domains, which is efficiently implemented with DCLS. Hence, at least some cores in the safety island need to implement DCLS. The SafeLS realizes DCLS for NOEL-V cores in the SELENE SoC~\cite{SafeLS}. However, as explained before, DCLS is generally expensive if not needed for some tasks since redundant cores are not user visible. Hence, different flavors of diverse redundancy can be deployed providing different tradeoffs, such as allowing cores to be used independently, although failing to provide diverse redundancy for I/O code. This is the case of the Safe Diversity Monitor (SafeDM) module~\cite{SafeDM}, which allows measuring whether diversity exists across two cores. Conversely, the Safe Diversity Enforcement (SafeDE)~\cite{SafeDE} module allows enforcing some time staggering, and hence, diversity across two cores running a task redundantly. The SafeSoftDR software module~\cite{SafeSoftDR} could be used instead since it provides the same functionality as SafeDE in a less efficient manner but without requiring any hardware support. A comparison across the different mechanisms can be found in~\cite{SafeDX}.

Note that, DCLS is intrinsically highly coupled with the redundant cores, and hence, only available for the safety island. Instead, SafeDE, SafeDM and SafeSoftDR can manage diversity for non-DCLS cores. Therefore, they can be tailored to deliver diverse redundancy to cores in the HPC island from the safety island.


\subsection{Diverse Redundancy for Accelerators}
Full redundancy for accelerators such as GPUs is generally not present in HPC devices. Therefore, it is not possible orchestrating diverse redundancy across multiple accelerator instances as done for cores with SafeDE and SafeSoftDR. 

However, accelerators are often highly parallel and offer large internal redundancy this has been leveraged in some works to implement some form of diverse redundancy with appropriate software and hardware support~\cite{divredINTEL,divredNVIDIA}. This type of support can be potentially integrated in the safety island, which can, for instance, offload redundant kernels in a GPU of the HPC island inducing diversity with different means (e.g., intrinsics support, scheduling policy characteristics).

In the context of Deep Neural Networks (DNNs), high -- yet not perfect -- accuracy rates are obtained for processes such as object detection and classification. DNNs often rely on approximation and stochastic behavior, and hence, do not generally require bit-level precision. Instead, high -- yet not full -- precision is wanted at semantic level (e.g., properly detecting and classifying an object) regardless of whether the accuracy is a bit higher or lower. In that context, it is possible deploying lower-cost and approximate (e.g., using lower precision arithmetic) accelerators in the safety island providing diverse redundancy to large and precise accelerators in the HPC island as long as the former are capable of detecting large deviations for the predictions of the latter~\cite{SAURIA}. Such DMR scheme has already been realized in \cite{TRUST}.


\subsection{Watchdogs}
As part of the architectural design of safety-related functionalities, watchdogs are popular since they allow checking the aliveness of specific components. At hardware level, watchdogs are also popular and, in the context of the safety island, they can be deployed to monitor the aliveness of specific components in the safety island as well as in the HPC island (or the complete HPC island). 

Generally, watchdogs are expected to be made sufficiently independent of the item being monitored, e.g., with independent clock and power supply. Hence, this is expected to hold by construction in the case of loose integration of the safety island. However, specific design rules must be followed for both coupled integration of the safety and HPC islands, and watchdogs monitoring components part of the safety island itself.

Watchdogs can monitor clock signals, cycle counters, instruction counters, or time-to-response for some devices. For instance, one could couple a watchdog to the SafeTI so that the latter sends a request requiring response to a specific component in the HPC island, while the watchdog awaits for an answer within a specific time bound. If such response does not arrive timely, the watchdog may raise an interrupt to be captured by the system software in the safety island.


\subsection{Virtualization Extensions}
Hypervisors and RTOSs, often required in safety-critical systems, require appropriate virtualization capabilities to offer partitioning services to the guest operating system. Such virtualization can only be realized if supported by the hardware platform. Hence, virtualization extensions become mandatory for the safety island. For instance, the aforementioned NOEL-V cores in the SELENE SoC implement such extension and have been proven effective to run hypervisors on top, such as fentISS' XtratuM~\cite{SCC,xtratum}.


\subsection{Logging Support}
Most HPC devices include some form of tracing support. However, information traced can be abundant and produced continuously, which, in general, requires a host computer to process it. Such information can be of much use to diagnose the source of some errors, or, at least, to enable reproducibility for diagnostics purposes. 
The safety island can act as such host computer. However, storage capabilities are limited for the safety island (e.g., typically KBs for on-chip storage and MBs for off-chip storage), and hence, either information is dropped by, for instance, retaining the most recent traced information only, or summarized in the form of logs. Retaining recent information can be easily done with trace buffers where information is stored using a FIFO policy. Logging requires, instead, a tradeoff between the details retained and the hardware cost to retain them. The higher the degree of information loss, the lower the cost to store remaining information. For instance, one can track timestamps for specific events, which would require large storage capabilities or restricting trace recording to a limited time window. Alternatively, one could use counters to track occurrences of those events -- potentially broken down across multiple categories, with much lower storage cost, but losing information relative to the timing of those events.
For instance, some authors proposed an error logger for caches tracking error location to diagnose permanent faults~\cite{LogCacheFaults}. 

\subsection{Chiplet Integration Technologies} 
In case of a loose integration with a chiplet-based safety island, Universal Chiplet Interconnect Express (UCIe) comes as the standardizing solution to die-to-die interconnectivity. The layered protocol specifies a die-to-die adapter layer and a protocol layer, the latter supporting PCIe or CXL, with further protocol mappings planned. This requires, however, that communicating chiplets adhere to standards. For instance, UCIe’s specification does not cover packaging/bridging technology used to provide the physical link between chiplets. It is bridge-agnostic, meaning chiplets can be linked via different mechanisms such as fanout bridge, silicon interposers (i.e. 2.5D packaging) or other packaging technologies such as 3D packaging. Nevertheless, standards such as bump pitch, must be taken into account, meaning RISC-V platforms would require dedicated, standardized support for UCIe, which could potentially hinder observability.

In terms of packaging technology, for instance, Intel’s EMIB (Embedded Multi-Die Interconnect Bridge) is a 2.5D packaging technique used to connect dies on the same substrate. 2.5D refers to the integration of dies/chiplets on a substrate using an interposer. It brings specific advantages such as larger die count and larger package configurations, lower cost than full size silicon interposer and support for high data rate signaling between adjacent die.

3D packaging is an alternative to interposers. 3D packaging refers to the direct high-density interconnection of chips through TSV (through-silicon via).
In 3D packaging, chiplets are placed on top of one another instead of horizontally next to one another, forming a 3D structure with each chiplet occupying a layer. Finally, the interposer connects the 3D assembly of the chiplet with the substrate. For instance, Foveros is a high-performance 3D packaging technology. 




\section{Other Applications}
\label{sec:appl}

The safety island can be used, for obvious reasons, in contexts where system requirements include the combination of high-performance needs and safety requirements. However, most of its features can provide other type of services to HPC devices, such as, security and Reliability, Availability and Serviceability (RAS). Therefore, there is a broad area of application for the safety island beyond safety requirements strictly. For the sake of illustration, we introduce the use of the safety island for RAS and security applications in the remaining of this section.

\subsection{RAS}
HPC devices used for servers and supercomputers, to name some application domains, have strict RAS requirements. This relates to the relatively higher criticality of the applications run in those domains when compared with most desktop computers, as well as the much higher exposure to faults due to having very high occupancy and, in the context of supercomputers, typically thousands of computing nodes operating in parallel cooperatively. For instance, in the case of a supercomputer where parallel applications required on average 1,000 CPUs, acceptable failure rates would decrease at least by a factor of 1,000 with respect to single-CPU computers.

Components such as the SafeSU or watchdogs can provide error detection capabilities able to trigger recovery actions at software level. Analogously, logging features can be used to assist recovery by providing relevant information about the error detected, or can also be used to anticipate unrecoverable errors by monitoring recoverable ones. For instance, permanent faults can be detected by diagnosing error location with specific logging capabilities so that appropriate actions (e.g., CPU replacement in a supercomputer) can be taken before permanent faults lead to any unrecoverable error.


\subsection{Security}
It is well-known that all safety-critical systems are also security-critical. This relates to the fact that unintended failures in safety-critical systems could be produced intendedly by an attacker, hence creating at least similar risks. The opposite, instead, is not true, and systems can be security-critical but not safety-critical (e.g., a system managing personal information). 

A subset of the security concerns have analogous behavior to that of the safety concerns, being the only difference whether their root cause is intended (security concerns) or unintended (safety concerns). For instance, abuse in the access to shared resources may be caused by a faulty application or by a malicious attack. In both cases, the SafeSU could be leveraged to, at least, detect the abuse and take corrective actions. Tracing and logging features could be used to discern intended from unintended attacks based on their history or frequency of occurrence, for instance.

Security concerns may often require countermeasures to stop or fool attacks. Components such as the SafeTI may be leveraged for that purpose generating traffic that degrades the ability of the attacker to deduce information from the victim. For instance, in the case of side-channel attacks learning from memory access patterns of the victim, traffic can be injected making the attacker believe such traffic belongs to the victim so that wrong conclusions are reached (e.g., obtaining a wrong key or failing to narrow down enough the possibilities for the victim's key).

In any case, if the safety island is also used for security purposes, a number of additional security-specific technologies may need to be added to the safety (now safety-and-security) island, such as cryptographic accelerators, {\color{black}and means for detecting and defeating attacks.
Note that those security-related features focus on providing security services to the HPC device, whereas those discussed in Section~\ref{sec:seccons} focus on making safety island operation secure.}



\section{Conclusions}
\label{sec:concl}

There is an increasing need for the use of HPC devices in safety-critical systems, but those devices lack enough controllability and observability channels, as well as adequate support to realize key safety measures. Hence, solutions are required to enable the safe use of HPC devices.

This paper presents our concept of a safety island and its main constituents to enable the safe use of HPC devices for safety-critical systems. In particular, we analyzed some key tradeoffs related to the degree of coupling of the safety and HPC islands, identified key components needed or highly convenient to have in the safety island to fulfill its due, and assessed some other types of applications of the safety island with overlapping needs, such as applications with RAS and/or security requirements.



\section*{Acknowledgements}
{\color{black}
BSC authors contribution is part of the project ({\color{black}ISOLDE}), funded by MCIN/AEI/10.13039/501100011033 and the European Union NextGenerationEU/PRTR, and the European Union's Horizon Europe Programme under project KDT Joint Undertaking (JU) under grant agreement No 101112274. This work has also been partially supported by the Spanish Ministry of Science and Innovation under grant PID2019-107255GB-C21 funded by MCIN/AEI/10.13039/501100011033.
}

\bibliographystyle{IEEEtran}
\bibliography{biblio}



\end{document}
