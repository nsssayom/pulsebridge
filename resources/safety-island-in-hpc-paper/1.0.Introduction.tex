\section{Introduction}
\label{sec:intro}

HPC processors can deliver the performance needed for future embedded systems of autonomous cars and aircraft, which will rely heavily on performance-hungry Artificial Intelligence (AI) software, as well as a high number of processes to manage simultaneous events triggered by a plethora of sensors and by timers periodically. However, the support that those devices offer for their use in safety-related applications is different across devices, and so are the guarantees or additional support they need -- if any -- for their use in applications with varying needs in terms of integrity level, fail-safe or fail-operational needs, performance predictability, and time to recover from different types of errors.
 

In general, HPC processors can be used for fail-safe applications as long as a high-integrity microcontroller unit (MCU) is deployed along with the HPC processor, and the overall system architected so that the MCU can manage all errors affecting both, the HPC processor or the MCU itself, and guide the system to a safe state timely in accordance with application safety requirements. However, if the MCU cannot manage all errors in the HPC processor preserving safety requirements, or if the HPC processor must remain fault-tolerant to meet such safety requirements, then additional support is needed beyond that of the MCU. Such support can be part of the HPC device itself, or be delivered in the form of an enhanced MCU with extended safety capabilities to assist the HPC device.

To overcome the potential limitations that some HPC devices may offer for their use in safety-critical applications, \emph{safety islands} have been proposed recently~\cite{SiemensSI,IntelGo,NVIDIAdrive,NVIDIAdrive2}. While the term \emph{safety island} includes a heterogeneous set of devices, they normally offer two key sets of features: (1) a safe enclave to run safety-related applications, in the same way an MCU does. In fact, an MCU can be regarded as a safety island. However, safety islands offer lower performance than HPC devices, at least for a set of performance-demanding applications, and hence the need for an accompanying HPC device. (2) A safety island also offers specific support to manage an external device that needs to run safety-related functionalities. Such support may include watchdog services, performance monitoring capabilities, ability to initiate test capabilities in the other device or to test it externally, or support to orchestrate diverse redundant execution in the external device to name a few.

When considering the deployment of a safety island along with an HPC device, considerations about integration cannot be neglected. For instance, new cost-efficient chip production and packaging options (multi-die chips -- EMIB\footnote{Embedded Multi-Die Interconnect Bridge}, Foveros~\cite{packagingIntel}) provide a solution to deploy separate dies in the same package, which benefits a number of safety island requirements like fault independence with different dies while preserving high bandwidth communication.

This paper presents our own concept of a safety island, as well as our strategy to realize it based on open source RISC-V~\cite{RISCV} components with the aim of easing its use and adoption. 
In our case, we aim at providing a rich set of safety services to the HPC device to enable the deployment of safety-relevant applications on top with varying safety requirements, despite the potentially limited support that the HPC device may offer natively. Those services foreseen for our safety island include the following:
\begin{itemize}
\item Controllability features to configure the HPC device for fault management and containment, and for providing predictable performance.
\item Observability features for system validation, error diagnosis, and as the basis to build informed safety measures on top.
\item Safety measures to contain and manage errors, and to mitigate abnormal performance conditions.
\end{itemize}

In this paper we provide the following contributions for the design and deployment of our view of a safety island:
\begin{enumerate}
\item We present our concept of a safety island along with examples for its practical realization focusing on open source SoCs based on the RISC-V ISA.
\item We identify existing open source components providing controllability and observability features, and safety measures, appropriate for our Safety Island.
\item We identify further features to be developed in the future to complete and complement our safety island.
\item We devise further applications of the Safety Island beyond safety, such as security, reliability, and power and temperature management. To some extent, the foreseen security extensions are comparable to the ``Security Island'' already offered by some processors (e.g., Arm's TrustZone~\cite{TrustZone}). 
\end{enumerate}

The rest of the paper is organized as follows. Section~\ref{sec:back} provides some background and other solutions comparable or related to the safety island proposed in this paper. Section~\ref{sec:sais} presents our concept of safety island. Section~\ref{sec:comp} describes convenient components for our safety island, whether they already exist or not. Section~\ref{sec:appl} shows how the safety island could be used for other applications. Finally, Section~\ref{sec:concl} summarizes this paper.



